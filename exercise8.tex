% author: Tomas Trnka
% mail: tomas@trnkatomas.eu
% date: 2013-07-04

\documentclass[a4paper,10pt]{article}
%\usepackage[czech]{babel}
%\usepackage[T1]{fontenc}
\usepackage[hmargin=2.2cm,vmargin=2.2cm]{geometry}
\usepackage[utf8x]{inputenc}
\usepackage{fancyhdr}
\usepackage{amsmath, tabu} 
\usepackage{enumerate}
\usepackage{dsfont}
\usepackage[hang,small,bf]{caption}    % fancy captions
\usepackage{tikz}	
\usetikzlibrary{backgrounds,fit,decorations.pathreplacing}  % TikZ libraries
\newcommand{\bra}[1]{\ensuremath{\left\langle#1\right|}} % Dirac Bra
\newcommand{\ket}[1]{\ensuremath{\left|#1\right\rangle}} % Dirac Kets
\usepackage{hyperref}
\pagestyle{fancy}
\headheight 15pt
\lhead{QIP, Fall 2014}
\rhead{Tomas Trnka}
\newcommand{\set}[1]{\ensuremath{\left\lbrace #1 \right\rbrace}}
\newcommand{\role}[1]{\ensuremath{\left\langle #1 \right\rangle}}
\newcommand{\cara}{\begin{center}\rule{140mm}{.2mm}\end{center}}
\newcommand{\mI}{\ensuremath{^\mathcal{I}}}
\newcommand{\Tbox}[1]{\ensuremath{\mathcal{T}}-Box#1}
\newcommand{\Abox}[1]{\ensuremath{\mathcal{A}}-Box#1}
\newcommand{\mC}[1]{\ensuremath{\mathcal{#1}}}
\newcommand{\Tc}{\ensuremath{\mathcal{T}_c}}
\newcommand{\qb}[1]{\ensuremath{\vert{#1}\rangle}}
\newcommand{\al}{\ensuremath{\alpha}}
\newcommand{\asp}{\ensuremath{\frac{1}{\sqrt{2}}}}
\newcommand{\ap}{\ensuremath{\frac{1}{2}}}
\begin{document}
\section*{QIPexercises -- Exercise 9}
\begin{enumerate}[1.]
\item Because they both share the key, B can simply apply to the state that he obtains, let's denote it as:
$$
\ket{\varphi}=X^{b_0}Z^{b_1}\ket{\psi}
$$
the $X$ and $Z$ controlled with the shared key in reverse order. He can do it because $X$ and $Z$ are inversions of themselves:
$$
Z^{b_1}X^{b_0}\ket{\varphi}=Z^{b_1}X^{b_0}X^{b_0}Z^{b_1}\ket{\psi}=
Z^{b_1}Z^{b_1}\ket{\psi}=\ket{\psi}
$$

\item For computing the  density matrix we have to compute the output for the encryption circuit.
Just to commemorate how $X$ and $Z$ looks like
$$
X = 
\left[ \begin{array}{cc} 
0 & 1 \\
1 & 0
\end{array} \right]
\text{, }Z = \left[ \begin{array}{cc} 
1 & 0 \\
0 & -1
\end{array} \right]
$$

Applying $Z^{b_1}$ and $X^{b_0}$ we obtain:
$$
\alpha^{b_0} \left( -1^{b_1}\beta \right)^{1-b_0}\ket{0} + \alpha^{1-b_0} \left( -1^{b_1}\beta \right)^{b_0}\ket{1}
$$

Since the $b_0$ and $b_1$ can be only $0$ or $1$ zero we can determine all possible outcomes:
$$
\begin{tabu}{|c|c|r|}\hline
  b_0 & b_1 & \text{output} \\ \hline
  0 & 0 & \beta\ket{0} + \alpha\ket{1} \\ \hline
  1 & 0 & \alpha\ket{0} - \beta\ket{1} \\ \hline
  0 & 1 & -\beta\ket{0} + \alpha\ket{1} \\ \hline
  1 & 1 & \alpha\ket{0} + \beta\ket{1} \\ \hline
\end{tabu}
$$

That is something what we have already seen in the quantum transportation algorithm.
All these states have the same probability, so we can create ensemble
$$
{(\frac{1}{4},\beta\ket{0} + \alpha\ket{1}),
 (\frac{1}{4},\alpha\ket{0} - \beta\ket{1}),
 (\frac{1}{4},-\beta\ket{0} + \alpha\ket{1}),
 (\frac{1}{4},\alpha\ket{0} + \beta\ket{1})}
$$
Now we can create density matrix as:
$$
\frac{1}{4}\left(
\left(
\begin{array}{c}
\beta\\
\alpha
\end{array}
\right)
\left(
\begin{array}{cc}
\beta^* & \alpha^*
\end{array}
\right)
+
\left(
\begin{array}{c}
\alpha\\
-\beta
\end{array}
\right)
\left(
\begin{array}{cc}
 \alpha^*& -\beta^*
\end{array}
\right)
+
\left(
\begin{array}{c}
-\beta\\
\alpha
\end{array}
\right)
\left(
\begin{array}{cc}
-\beta^* & \alpha^*
\end{array}
\right)
+
\left(
\begin{array}{c}
\alpha\\
\beta\\
\end{array}
\right)
\left(
\begin{array}{cc}
\alpha^* & \beta^*
\end{array}
\right)
\right)
$$

$$
\frac{1}{4}\left(
\left(
\begin{array}{cc}
||\beta||^2 & \alpha^*\beta \\
\alpha\beta^* & ||\alpha||^2
\end{array}
\right)
+
\left(
\begin{array}{cc}
||\alpha||^2 & \alpha^*-\beta \\
-\alpha^*\beta &  ||\beta||^2
\end{array}
\right)
+
\left(
\begin{array}{cc}
||\beta||^2 & \alpha^*-\beta \\
-\alpha\beta^* &  ||\alpha||^2
\end{array}
\right)
+
\left(
\begin{array}{cc}
||\alpha||^2 & \alpha^*\beta \\
\alpha^*\beta & ||\beta||^2
\end{array}
\right)
\right)
$$
$$
\frac{1}{4}\left(
\begin{array}{cc}
2(||\alpha||^2 + ||\beta||^2) & 0 \\
0 & 2(||\alpha||^2 + ||\beta||^2)
\end{array}
\right)
=
\frac{1}{4}\left(
\begin{array}{cc}
2 & 0 \\
0 & 2
\end{array}
\right)
=
\left(
\begin{array}{cc}
\frac{1}{2} & 0 \\
0 & \frac{1}{2}
\end{array}
\right)
$$

And from this density matrix the observer know nothing, when he do not know the secret key, it appears as completely random state.

\item For one random control bit and one shared I would suggest to do:
$$
Y^bZ^r \ket{\psi}
$$

We proceed through the same procedure as above and ended up with the probability table
$$
\begin{tabu}{|c|c|r|}\hline
  r & b & \text{output} \\ \hline
  0 & 0 & i\beta\ket{0} - i\alpha\ket{1} \\ \hline
  1 & 0 & -i\beta\ket{0} - i\alpha\ket{1} \\ \hline
  0 & 1 & i\alpha\ket{0} + i\beta\ket{1} \\ \hline
  1 & 1 & i\alpha\ket{0} - i\beta\ket{1} \\ \hline
\end{tabu}
$$

The when we compute the density matrix:
$$
\frac{1}{4}\left(
\left(
\begin{array}{c}
i\beta\\
-i\alpha
\end{array}
\right)
\left(
\begin{array}{cc}
i\beta & -i\alpha
\end{array}
\right)
+
\left(
\begin{array}{c}
-i\beta\\
-i\alpha
\end{array}
\right)
\left(
\begin{array}{cc}
-i\beta&-i\alpha
\end{array}
\right)
+
\left(
\begin{array}{c}
i\alpha\\
i\beta
\end{array}
\right)
\left(
\begin{array}{cc}
i\alpha & i\beta
\end{array}
\right)
+
\left(
\begin{array}{c}
i\alpha\\
-i\beta\\
\end{array}
\right)
\left(
\begin{array}{cc}
i\alpha & -i\beta
\end{array}
\right)
\right)
$$

$$
\frac{1}{4}\left(
\left(
\begin{array}{cc}
-||\beta||^2 & \alpha\beta \\
\alpha\beta & -||\alpha||^2
\end{array}
\right)
+
\left(
\begin{array}{cc}
-||\beta||^2 & -\alpha-\beta \\
-\alpha\beta &  -||\alpha||^2
\end{array}
\right)
+
\left(
\begin{array}{cc}
-||\alpha||^2 & -\alpha\beta \\
-\alpha\beta &  -||\beta||^2
\end{array}
\right)
+
\left(
\begin{array}{cc}
-||\alpha||^2 & \alpha\beta \\
\alpha\beta & -||\beta||^2
\end{array}
\right)
\right)
$$
$$
-\frac{1}{4}\left(
\begin{array}{cc}
2(||\alpha||^2 + ||\beta||^2) & 0 \\
0 & 2(||\alpha||^2 + ||\beta||^2)
\end{array}
\right)
=
-\frac{1}{4}\left(
\begin{array}{cc}
2 & 0 \\
0 & 2
\end{array}
\right)
=
-\left(
\begin{array}{cc}
\frac{1}{2} & 0 \\
0 & \frac{1}{2}
\end{array}
\right)
$$

This is the same as previous, the attacker knows nothing what was sent, this also work just because we are working with assumption that $\alpha$ and $\beta$ are real.

When Bob would like to decode, he firstly apply the $Y$ gate which is also its own inverse. And then he can flip the coin whether he will use $Z$ gate or not. This gate will influence only the sign of the \ket{1} coefficient which we can consider once again as a `unimportant phase shift'.
\end{enumerate}
\end{document}