% author: Tomas Trnka
% mail: tomas@trnkatomas.eu
% date: 2013-07-04

\documentclass[a4paper,10pt]{article}
%\usepackage[czech]{babel}
%\usepackage[T1]{fontenc}
\usepackage[hmargin=2.2cm,vmargin=2.2cm]{geometry}
\usepackage[utf8x]{inputenc}
\usepackage{fancyhdr}
\usepackage{amsmath, tabu} 
\usepackage{enumerate}
\usepackage{dsfont}
\usepackage[hang,small,bf]{caption}    % fancy captions
\usepackage{tikz}	
\newcommand{\bra}[1]{\ensuremath{\left\langle#1\right|}} % Dirac Bra
\newcommand{\ket}[1]{\ensuremath{\left|#1\right\rangle}} % Dirac Kets
\usepackage{hyperref}
\pagestyle{fancy}
\headheight 15pt
\lhead{QIP, Fall 2014}
\rhead{Tomas Trnka}
\newcommand{\set}[1]{\ensuremath{\left\lbrace #1 \right\rbrace}}
\newcommand{\role}[1]{\ensuremath{\left\langle #1 \right\rangle}}
\newcommand{\cara}{\begin{center}\rule{140mm}{.2mm}\end{center}}
\newcommand{\mI}{\ensuremath{^\mathcal{I}}}
\newcommand{\Tbox}[1]{\ensuremath{\mathcal{T}}-Box#1}
\newcommand{\Abox}[1]{\ensuremath{\mathcal{A}}-Box#1}
\newcommand{\mC}[1]{\ensuremath{\mathcal{#1}}}
\newcommand{\Tc}{\ensuremath{\mathcal{T}_c}}
\newcommand{\qb}[1]{\ensuremath{\vert{#1}\rangle}}
\newcommand{\al}{\ensuremath{\alpha}}
\newcommand{\asp}{\ensuremath{\frac{1}{\sqrt{2}}}}
\newcommand{\aps}{\ensuremath{\frac{1}{2\sqrt{2}}}}
\newcommand{\ap}{\ensuremath{\frac{1}{2}}}
\begin{document}
\section*{CSS codewords -- Exercise 12}
We are supposed to show that the codewords states for the CSS codewords \ket{\xi_{x,z,v_k}} satisfy:
$$
\frac{1}{\sqrt{2^n}}\sum_{j\in \{0,1\}^n} \ket{j}\ket{j} = \frac{1}{\sqrt{2^n}}\sum_{v_k \in  C_{1/2},x\in C_{all/1},z \in C_{all/2^\bot}}\ket{\xi_{x,z,v_k}}\ket{\xi_{x,z,v_k}}
$$
\begin{enumerate}[1.]
\item 
The CSS codewords can be rewritten as:
$$
\ket{\xi_{x,z,v_k}} = \frac{1}{\sqrt{|C_2|}} \sum_{y \in C_2} (-1)^{y \cdot z} \ket{v_k + x + y},
$$
and then we obtain the whole right part of equation as:
$$
\sum_{v_k \in  C_{1/2},x\in C_{all/1},z \in C_{all/2^\bot}}\frac{1}{\sqrt{|C_2|}} \sum_{y \in C_2} (-1)^{y \cdot z} \ket{v_k + x + y}\frac{1}{\sqrt{|C_2|}} \sum_{y \in C_2} (-1)^{y \cdot z} \ket{v_k + x + y}
$$
We can reorganize the sums:
$$
\frac{1}{|C_2|}\sum_{z \in C_{all/2^\bot}}\sum_{y \in C_2} (-1)^{2^{(y \cdot z)}}
\sum_{v_k \in  C_{1/2},x\in C_{all/1}} \ket{v_k + x + y}\ket{v_k + x + y}
$$

As we have seen last week, when we choose $v_k$ from different $x$ then where the $v_k$ belongs we obtain zeros, ie. they are orthogonal and create a basis. So from this we can conclude to obtain valid states, we can choose all $x$ in $C_{all/1}$ where $|C_{all/1}| = 2^{n-k_1}$. And for every $x$ we can choose corresponding $v_k$ that belongs to the particular $x$ to obtain the basis for the coset of x. We have $|C_{1/2}|=2^{k_1-k_2}$ possibilities.

Let's take a look on the first sums, we iterate with $y$ over $C_2$ which gives us another $|C_2|=2^{k_2}$ possibilities. But we have to take into account the hint that for code iterating over the cosets the result will be $\sum{i}(-1)^{y*z_i} = 0$. This is because in half of the cases it remains -1 and in the second half it becomes 1.
I'm convinced that because we now have something different, more specific $-1^2=1$ and that the whole sum become $1^{zy}$ which is always 1.

Since we can choose $|C_{all/2^\bot}|=|C_2|$ it will cross out with the constant $\frac{1}{|C_2|}$ which is before the whole equation.

Now if we put together all possibilities we have:
$$
2^{n-k_1}\cdot 2^{k_1-k_2}\cdot 2^{k_2} = 2^{n-k_1+k_1-k_2+k_2} = 2^n
$$

So we have as many possibilities for the state as we have on the left side of original equation and we know that in binary field it is possible to encode $2^n$ states with $n$ bit string, and since all these possibilities are equiprobable we have to prepend the factor
$$
\frac{1}{\sqrt{2^n}}
$$
to obtain a valid state. Now have basis for the whole space on both sides of the equation and therefore the equality holds.

\end{enumerate}
\end{document}